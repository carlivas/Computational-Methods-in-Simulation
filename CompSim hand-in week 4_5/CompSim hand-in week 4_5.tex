\documentclass[acmtog]{acmart}
\usepackage{lipsum}
\usepackage{physics}

\usepackage{caption}
\usepackage{subcaption}

\usepackage{titlesec}
\usepackage{graphicx}
\usepackage{xcolor}

\newcommand{\draft}[1]{\noindent\textcolor{red}{\texttt{#1}}}

\renewcommand{\thesubsubsection}{--- \hspace{2pt} Step \arabic{subsubsection}: \hspace{-7pt}}
\titlespacing*{\subsubsection}{0.0\baselineskip}{2\baselineskip}{1\baselineskip}


\setcopyright{none} 
\makeatletter
\let\@authorsaddresses\@empty
\makeatother
\settopmatter{printacmref=false}
\renewcommand\footnotetextcopyrightpermission[1]{}
\AtBeginDocument{%
  \providecommand\BibTeX{{%
    Bib\TeX}}}
\begin{document}

\title{CompSim hand-in week 4/5}
\author{Carl Ivarsen Askehave}
\affiliation{
  \institution{(wfq585)}
  \country{University of Copenhagen}
}

\maketitle
\thispagestyle{empty}
\section{Week 4}
\subsection{Derivation of the 1D FEM problem and relevant terminology}
Given a governing equation
%
\begin{equation}
  \frac{ \partial^{2} y }{ \partial x^{2} } = c(x),
\end{equation}
%
with Dirichlet boundary conditions
%
\begin{equation}
  y(x_1) = a, \quad \text{and} \quad y(x_n) = b,
\end{equation}
%
the FEM is done in the 5 following steps:

\subsubsection{Rewriting into a volume integral}
Usually our governing equation is a differential equation, which can be
rewritten as a volume integral using the divergence theorem. This is done by
multiplying the equation with a test function and integrating over the domain.
We first introduce a trial function $v(x)$ that is 0 on the boundaries and
random inside.
%
\begin{align}
  v(x) & = 0 \quad \text{for} \quad x = x_1 \ \text{or} \ x_n                 \\
  v(x) & = \mathrm{random}, \quad \text{for} \quad x \in \ ]x_1, \dots, x_n[.
\end{align}
%
We then rewrite
%
\begin{equation}
  v(x) \left( \frac{ \partial^{2} y }{ \partial x^{2} }  - c(x) \right) = 0,
\end{equation}
which is trivially true on the boundary, since $v(x) = 0$, but since we know that $v(x)$ is not 0 inside the domain, we know that the quantity in the parentheses must be $\frac{ \partial^{2} y }{ \partial x^{2} } - c(x) = 0$.
%
We now take the integral of both sides
%
\begin{equation}
  \int_{x_1}^{x_n} v(x) \left( \frac{\partial^{2} y }{ \partial x^{2} } - c(x) \right) \, dx = 0.
\end{equation}
%
We do something similar for both the boundary conditions, introducing two new
trial functions
%
\begin{equation}
  w_1(x) = \begin{cases}
    0               & \text{for} \ x = x_1, \\
    \mathrm{random} & \text{otherwise},
  \end{cases}
\end{equation}
%
and
%
\begin{equation}
  w_n(x) = \begin{cases}
    0               & \text{for} \ x = x_n, \\
    \mathrm{random} & \text{otherwise},
  \end{cases}
\end{equation}
%
such that
%
\begin{equation}
  \int_{x_1}^{x_n} w_1 \left( y(x_1) - a \right) \, dx = 0, \quad \mathrm{and} \quad \int_{x_1}^{x_n} w_b \left( y(x_n) - b \right) \, dx = 0.
\end{equation}

\subsubsection{Integrating by parts}
Using the rule for integration by parts, we can rewrite the volume integral as
%
\begin{align}
  \int_{x_1}^{x_n} v(x)\frac{ \partial^{2} y }{ \partial x^{2} }\,dx - \int_{x_1}^{x_n} v(x) c(x) \, dx                                                                                         & = 0, \\
  \left[v \, \frac{ \partial y }{ \partial x } \right]_{x_1}^{x_n} - \int_{x_1}^{x_n} \frac{ \partial v }{ \partial x } \frac{ \partial y }{ \partial x } \, dx - \int_{x_1}^{x_n} v \, c \, dx & = 0, \\
  \int_{x_1}^{x_n} \frac{ \partial v }{ \partial x } \frac{ \partial y }{ \partial x } \, dx + \int_{x_1}^{x_n} v \, c \, dx                                                                    & = 0,
\end{align}
%
where we've used that the trial function is 0 at the boundaries, i.e. $v(x_1) =
  v(x_n) = 0$. The main point of doing this integration by parts step, is to
reduce the condition of twice differentiability on $y$ to once
differentiability, and we see that there are only first order derivatives in
our final expression, so we're happy. One thing to note, is that we end up
having a derivative of our trial function $v(x)$ which we didn't have before.

\subsubsection{Make an approximation}
We make the FEM approximation of $\boldsymbol y$
%
\begin{equation}
  y(x) \approx \tilde y(x) = \sum_{i=0}^n N_i(x) \hat{y}_i,
\end{equation}
%
which is equivalent to the matrix expression
%
\begin{equation}
  \tilde{y}(x) = \boldsymbol N(x) \boldsymbol{ \hat{y}} =\begin{bmatrix}
    N_1(x), \ N_2(x), \ \dots, \ N_n(x)
  \end{bmatrix} \begin{bmatrix}
    \hat{y}_1 \\
    \hat{y}_2 \\
    \vdots    \\
    \hat{y}_n
  \end{bmatrix}.
\end{equation}
%
The functions $N_i(x)$ are called the \textit{shape} or \textit{basis
  functions}, and are defined to follow the conditions
%
\begin{equation}
  N_i(x) = \begin{cases}
    1 \quad \text{for} \quad x= x_i, \\
    0 \quad \text{otherwise},
  \end{cases} \quad \mathrm{and} \quad \sum_i N_i(x) = 1,
\end{equation}
%
the second of which is the condition for \textit{partition of units} holds
which says that the sum of the basis functions $N_i(x)$ at any specific point
$x$ (not necessarily on our grid points $x_i$) always equals unity. A common
choice of basis functions are the piecewise linear functions
%
\begin{align}
  N_i (x) & = \max\left( 0, 1- \left\lvert  \frac{x_i - x}{\Delta x}  \right\rvert \right)                                       \\
          & = \begin{cases}
                \displaystyle1- \left\lvert  \frac{x_i - x}{\Delta x}  \right\rvert & \text{for} \quad x_{i-1} \leq x \leq x_{i+1} \\
                0                                                                   & \mathrm{otherwise.}
              \end{cases}
\end{align}
%
Now we want to input our approximation $\tilde{y}(x) = \boldsymbol N(x)
  \boldsymbol{\hat{y}}$ into our volume integral
%
\begin{align}
  \int_{x_1}^{x_n} \frac{ \partial v }{ \partial x } \frac{ \partial \tilde{y} }{ \partial x } \, dx + \int_{x_1}^{x_n} v \, c \, dx                                         & = 0, \\
  \int_{x_1}^{x_n} \left( \frac{ \partial v }{ \partial x } \frac{ \partial \boldsymbol N}{ \partial x } \right) \boldsymbol {\hat{y}} \, dx + \int_{x_1}^{x_n} v \, c \, dx & = 0.
\end{align}
%
Now, after we make a choice of trial function $v$ in a moment, the only thing
unknown in this equation is $\boldsymbol{\hat{y}}$, which we can then solve
for.
\subsubsection{Choose a \textit{trial function}}
One common choice for the trial function is using the basis functions, as such
%
\begin{equation}
  v(x) \equiv \boldsymbol N(x) \boldsymbol {\delta\hat{y}},
\end{equation}
%
where $\boldsymbol{\delta \hat{y}}$ are random values. This is called the
\textit{Galerkin method} . Inserting this into the volume integral gives us
%
\begin{align}
  \int_{x_1}^{x_n} \boldsymbol {\delta \hat{y}}^T\frac{ \partial \boldsymbol N^T }{ \partial x } \frac{ \partial \boldsymbol N}{ \partial x } \boldsymbol {\hat{y}} \, dx + \int_{x_1}^{x_n} \boldsymbol N \boldsymbol {\delta \hat{y}} \, c \, dx   & = 0, \\
  \boldsymbol {\delta \hat{y}}^T\left[\left(  \int_{x_1}^{x_n} \frac{ \partial \boldsymbol N^T }{ \partial x } \frac{ \partial \boldsymbol N}{ \partial x } \, dx \right) \boldsymbol {\hat{y}} + \int_{x_1}^{x_n} \boldsymbol N^T\, c \, dx \right] & = 0.
\end{align}
%
Now we know that the random variables $\boldsymbol{\delta \hat{y}}$ are
non-zero, so therefore we know that the quantity inside the parentheses must be
0:
%
\begin{equation}
  \left(  \int_{x_1}^{x_n} \frac{ \partial \boldsymbol N^T }{ \partial x } \frac{ \partial \boldsymbol N}{ \partial x } \, dx \right) \boldsymbol {\hat{y}} + \int_{x_1}^{x_n} \boldsymbol N^T\, c \, dx = 0.
\end{equation}
%
we realize that since the basis functions $\boldsymbol N$ are just bunch of
piecewise linear functions, their derivatives must be constant, and thus we can
easily evaluate the first term, which becomes a matrix $\boldsymbol K$ times
the functions $\boldsymbol{\hat{y}}$. We can also evaluate the second terms,
which we define to be the vector $- \mathbf f$. This means that we end up with
the linear system $\boldsymbol K \boldsymbol{\hat{y}} = \mathbf f$, where we
solve for the unknown $\boldsymbol{\hat{y}}$.
%
\begin{align}
  \underbrace{ \left(  \int_{x_1}^{x_n} \frac{ \partial \boldsymbol N^T }{ \partial x } \frac{ \partial \boldsymbol N}{ \partial x } \, dx \right) \boldsymbol {\hat{y}} }_{ \boldsymbol K \boldsymbol {\hat{y}} } + \underbrace{ \int_{x_1}^{x_n} \boldsymbol N^T\, c \, dx}_{ - \mathbf f } & = 0,             \\
  \boldsymbol K \boldsymbol {\hat{y}} - \mathbf f                                                                                                                                                                                                                                             & = \boldsymbol 0, \\
  \boldsymbol K \boldsymbol {\hat{y}}                                                                                                                                                                                                                                                         & = \mathbf f.
\end{align}
%

\subsubsection{Compute a solution}
Our linear system has the form
%
\begin{equation}
  \begin{bmatrix}
    K_{11} & \cdots & K_{1n} \\
    \vdots & \ddots & \vdots \\
    K_{n1} & \cdots & K_{nn}
  \end{bmatrix}
  \begin{bmatrix}
    \hat{y}_1 \\
    \vdots    \\
    \hat{y}_n
  \end{bmatrix} = \begin{bmatrix}
    f_1    \\
    \vdots \\
    f_n
  \end{bmatrix}
\end{equation}
%
Enforcing our boundary conditions $y(x_1) = a$ and $y(x_n) = b$ we get
%
\begin{equation}
  \begin{bmatrix}
    1            & 0            & \cdots & 0             \\
    K_{21}       & K_{22}       & \cdots & K_{2n}        \\
    \vdots       & \vdots       & \ddots & \vdots        \\
    K_{n-1, \,1} & K_{n-1, \,2} & \cdots & K_{n-1, \, n} \\
    0            & 0            & \cdots & 1
  \end{bmatrix}
  \begin{bmatrix}
    \hat{y}_1     \\
    \hat{y}_2     \\
    \vdots        \\
    \hat{y}_{n-1} \\
    \hat{y}_n
  \end{bmatrix} = \begin{bmatrix}
    a       \\
    f_2     \\
    \vdots  \\
    f_{n-1} \\
    b
  \end{bmatrix}
\end{equation}
%
where we've only modified the first and last rows of $\boldsymbol K$ and
$\boldsymbol{\hat{y}}$ (I'm not sure whether these are completely correct). We
call this the modified linear system
%
\begin{equation}
  \boldsymbol {K'} \boldsymbol {\hat{y}} = \mathbf{f'},
\end{equation}
%
and this can be solved. The exact matrix assembly of the boundary condition for
a 2D probem is a slightly more involved, but the idea is the same.

\subsection{Discussion of grid characteristics}
The choice of grid spacing is important in the FEM method. If we choose a grid
spacing that is too large, we might miss important features of the solution,
and if we choose a grid spacing that is too small, we might end up with a
system that is too large to solve.

The finite element method is powerful in the sense that it allows us to have a
non-uniform grid, but we have to be careful with this. The reason is, that, in
our formalism, the basis funcitons $\boldsymbol N^e$ are piecewise linear
functions, making our interpolated field values $\boldsymbol u(\boldsymbol x)$
linear approximations, which are dependent on the geometry on the elements.
Therefore, if we have a non-uniform grid, the quality and accuracy of our
solution might be poorer in some regions than others, and this might affect the
solution somewhere else in the domain.

This where the material from last week comes in. The quality of our basis
functions, and thus our linear approximations is directly related to the
regularity or quality of our mesh elements.

\subsection{Implementation and experiments}

\newpage
\section{Week 5}
\subsection{Derivation of FEM for a vector field}
The problem we want to look at is the deformation of a cantilever beam with due
to a traction force $\boldsymbol t$ and gravity $\boldsymbol g$. We define the
material (undeformed) coordinates $\boldsymbol X(t)$, the spatial (deformed)
coordinates $\boldsymbol x(t)$ and the displacement field
%
\begin{equation}
  \boldsymbol u (t) = \boldsymbol X(t) - \boldsymbol x(t).
\end{equation}
%
The governing equation of motion is \textit{Cauchy's equation of motion}
%
\begin{equation}
  \rho \ddot{\boldsymbol x} = \boldsymbol \nabla \cdot \boldsymbol \sigma + \boldsymbol b, \qquad \forall \, \boldsymbol x \in \Omega,
\end{equation}
%
where $\rho$ is the material density, $\boldsymbol \sigma$ is \textit{Cauchy's
  stress tensor} which describe the internal forces and $\mathbf b$ is the
external forces, e.g. gravity. All these quantities are parametrized by the
spatial coordinates $\boldsymbol x$. This equation holds for all control
volumes on inside of the domain $\Omega$. The condition for the boundary at
which we apply the traction force is given by \textit{Cauchy's stress
  hypothesis} as such
%
\begin{equation}
  \boldsymbol t = \boldsymbol \sigma \cdot \boldsymbol n, \qquad \forall  \, \boldsymbol x  \in \Gamma_{\boldsymbol t},
\end{equation}
%
where $\boldsymbol n$ is the vector normal to the surface. This equation holds
only for boundary at which $\boldsymbol t$ is applied $\Gamma_{\boldsymbol t}$.
The equation for our other boundary where we want the cantilever beam to be
fastened is a Dirichlet condition
%
\begin{equation}
  \boldsymbol x = \mathrm{constant} = \boldsymbol X, \qquad \forall \,\boldsymbol x \in \Gamma_D.
\end{equation}
%
Now we can apply the 5 steps of the FEM.
%
\subsubsection{Step 1: Rewrite to volume integral}
integral Now we can rewrite these conditions into a volume integral
%
\begin{equation}
  \int_{\Omega_{\boldsymbol x}} (\rho \ddot{\boldsymbol x} - \boldsymbol \nabla \cdot \boldsymbol \sigma - \boldsymbol b)^T \boldsymbol w \, d{\Omega_{\boldsymbol x}} = \boldsymbol 0.
\end{equation}
%
Now since our domain $\Omega_{\boldsymbol x}$ is going to change at each time
step of our simulation, and $\rho$, $\boldsymbol \sigma$ and $\boldsymbol b$
all are parametrized by our spatial (deformed) coordinates $\boldsymbol x$,
this integral will be a pain to solve. Therefore, we seek to change the
parametrization to be wrt. to the material coordinates $\boldsymbol X(t)$.
Therefore we do the transformation
%
\begin{equation}
  \boldsymbol x \to \boldsymbol X - \boldsymbol u.
\end{equation}
%
We now have
%
\begin{equation}
  \int_{\Omega_{\boldsymbol x}} (\rho \ddot{\boldsymbol x} - \boldsymbol \nabla \cdot \boldsymbol \sigma - \boldsymbol b)^T \boldsymbol w \,j\, d{\Omega_{\boldsymbol x}} = \boldsymbol 0,
\end{equation}
%
where $\boldsymbol w$ is a trial function (to be chosen later) and $j$ is the
determinant of the Jacobian matrix of our deformation field, which tells us about the volume change due to the change of
coordinates as such $d\Omega_{\boldsymbol x} = j \, d\Omega_{\boldsymbol x}$. Now,
we assume quasi-staticity $(\dot {\boldsymbol x} = \ddot {\boldsymbol x} =
  \boldsymbol 0)$ and small displacements $(\boldsymbol x \approx \boldsymbol X
  \Rightarrow j \approx 1)$, which gives us
%
\begin{equation}
  -\int_{\Omega_{\boldsymbol x}} (\boldsymbol \nabla \cdot \boldsymbol \sigma)^T \boldsymbol w \, d\Omega_{\boldsymbol x} - \int_{\Omega_{\boldsymbol x}} \boldsymbol b^T \boldsymbol w \, d \Omega_{\boldsymbol x} = \boldsymbol 0.
\end{equation}
%

\subsubsection{Step 2: Integration by parts}
The idea is to reduce the requirement of differentiability on the stress tensor $\boldsymbol \sigma$ and increase it on the trial function $\boldsymbol w$. Using the relation $- (\boldsymbol \nabla \cdot \boldsymbol \sigma)^T \boldsymbol w= \boldsymbol \sigma : \boldsymbol \nabla \boldsymbol w^T - \boldsymbol \nabla \cdot (\boldsymbol \sigma \boldsymbol w)$, we can rewrite
%
\begin{equation}
  \int_{\Omega_{\boldsymbol x}} \boldsymbol \sigma : \boldsymbol \nabla \boldsymbol w^T \, d\Omega_{\boldsymbol x} - \int_{\Omega_{\boldsymbol x}} \boldsymbol \nabla \cdot (\boldsymbol \sigma \boldsymbol w) \, d\Omega_{\boldsymbol x} - \int_{\Omega_{\boldsymbol x}} \boldsymbol b^T \boldsymbol w \, d\Omega_{\boldsymbol x} = \boldsymbol 0.
\end{equation}
%
We can rewrite the first term according to the identity $\boldsymbol \sigma :
  \boldsymbol \nabla \boldsymbol w^T = \boldsymbol \sigma : \frac{1}{2}
  (\boldsymbol \nabla \boldsymbol w + \boldsymbol \nabla \boldsymbol w^T) =
  \boldsymbol\sigma : \boldsymbol \epsilon_{\boldsymbol w}$, which holds because the
stress tensor is symmetric $\boldsymbol\sigma^T = \boldsymbol\sigma$. Now,
we're not quite finished with the second term, so we'll use [[Gauss' theorem]],
to rewrite it to

%
\begin{equation}
  \int_{\Gamma_{\boldsymbol X}} (\boldsymbol \sigma \boldsymbol w)^T \cdot\boldsymbol n \, d\Gamma_{\boldsymbol X} = \int_{\Gamma_{\boldsymbol X}} \boldsymbol w^T \boldsymbol \sigma^T \cdot\boldsymbol n \, d\Gamma_{\boldsymbol X} = \int_{\Gamma_{\boldsymbol X}} \boldsymbol w^T \boldsymbol \sigma \cdot\boldsymbol n \, d\Gamma_{\boldsymbol X} = \int_{\Gamma_{\boldsymbol t}} \boldsymbol w^T \boldsymbol t \, d\Gamma_{\boldsymbol t},
\end{equation}
%
where we've again used that the stress tensor is symmetric and furthermore,
we've used our traction boundary condition. Therefore our complete volume
integral becomes
%
\begin{equation}
  \underbrace{ \int_{\Omega_{\boldsymbol x}} \boldsymbol \sigma : \boldsymbol \epsilon_{\boldsymbol w} \, d\Omega_{\boldsymbol x} }_{ \boldsymbol P_e } \underbrace{\ - \int_{\Gamma_{\boldsymbol t}} \boldsymbol w^T \boldsymbol t \, d\Gamma_{\boldsymbol t} }_{ \boldsymbol P_t } \underbrace{\ - \int_{\Omega_{\boldsymbol x}} \boldsymbol w^T \boldsymbol b \, d\Omega_{\boldsymbol x} }_{ \boldsymbol P_b } = \boldsymbol 0.
\end{equation}
%
We have named the three different terms the \textit{power of elasticity} , the \textit{power
of traction}  and the \textit{power of body forces} . A curious side note is, that the
elasticity term is equal to the virtual work done against the elastic forces in
the material.

\subsection*{\textit{Physical definitions}}
\subsubsection*{Hooke's law for isotropic linear elastic materials}
Since we have assumed small displacements, we can use the strain energy density
function for isotropic linear elastic materials
%
\begin{equation}
  \psi = \mu \,\boldsymbol \epsilon : \boldsymbol \epsilon + \frac{\lambda}{2} \mathrm{tr}(\boldsymbol \epsilon)^2,
\end{equation}
%
where $\lambda$ and $\mu$ are the Lamé coefficients. Now we have
%
\begin{equation}
  \boldsymbol \sigma \equiv \frac{ \partial \psi }{ \partial \boldsymbol \epsilon } = 2 \mu \, \boldsymbol \epsilon + \lambda \,\mathrm{tr}(\boldsymbol \epsilon) \, \boldsymbol I,
\end{equation}
%
or equivalently
%
\begin{equation}
  \sigma_{ij} = 2 \,\mu \,\epsilon_{ij} + \lambda \,\delta_{ij} \sum_k \epsilon_{kk},
\end{equation}
%
where $\delta_{ij}$ is the \textit{kronecker-delta} . This is \textit{Hooke's law for
isotropic linear elastic materials}, which we can rewrite as a matrix equation
$\boldsymbol \sigma = \boldsymbol D \boldsymbol \epsilon$
%
\begin{equation}
  \underbrace{ \begin{bmatrix}
      \sigma_{xx} \\
      \sigma_{yy} \\
      \sigma_{zz} \\
      \sigma_{xy} \\
      \sigma_{xz} \\
      \sigma_{yz}
    \end{bmatrix} }_{ \boldsymbol \sigma } = \underbrace{ \begin{bmatrix}
      \lambda + 2 \mu & \lambda         & \lambda         & 0   & 0   & 0   \\
      \lambda         & \lambda + 2 \mu & \lambda         & 0   & 0   & 0   \\
      \lambda         & \lambda         & \lambda + 2 \mu & 0   & 0   & 0   \\
      0               & 0               & 0               & \mu & 0   & 0   \\
      0               & 0               & 0               & 0   & \mu & 0   \\
      0               & 0               & 0               & 0   & 0   & \mu
    \end{bmatrix} }_{ \boldsymbol D } \underbrace{ \begin{bmatrix}
      \epsilon_{xx}   \\
      \epsilon_{yy}   \\
      \epsilon_{zz}   \\
      2 \epsilon_{xy} \\
      2 \epsilon_{xz} \\
      2 \epsilon_{yz}
    \end{bmatrix} }_{ \boldsymbol \epsilon }.
\end{equation}
%

\subsubsection{Poisson ratio and Young's modulus}
The Lamé coefficients are described in terms of the \textit{Poisson ratio}  $\nu$ and
\textit{Young's modulus}  $E$, which are probably more intuitive to understand:
%
\begin{equation}
  \lambda = \frac{E \nu}{(1 + \nu)(1 - 2 \nu)}, \qquad \mu = \frac{E}{2(1+ \nu)}.
\end{equation}
%
Now we can rewrite $\boldsymbol D$
%
\begin{equation}
  \boldsymbol D = \frac{E}{(1+\nu)(1-2\nu)}\begin{bmatrix}
    d_0 & d_1 & d_1 & 0   & 0   & 0   \\
    d_1 & d_0 & d_1 & 0   & 0   & 0   \\
    d_1 & d_1 & d_0 & 0   & 0   & 0   \\
    0   & 0   & 0   & d_2 & 0   & 0   \\
    0   & 0   & 0   & 0   & d_2 & 0   \\
    0   & 0   & 0   & 0   & 0   & d_2
  \end{bmatrix},
\end{equation}
%
where $d_0 = 1- \nu$, $d_1 = \nu$ and $d_2 = (1-2\nu) / 2$.

\subsubsection{Rewriting the strain tensor}
The strain tensor $\varepsilon_{ij} = \frac{1}{2} ( \nabla_i u_j + \nabla_j
  u_i)$ can be rewritten as the product between an operator $\boldsymbol S$ and
our displacement field $\boldsymbol u$ as

%
\begin{equation}
  \boldsymbol \epsilon = \begin{bmatrix}
    \epsilon_{xx}   \\
    \epsilon_{yy}   \\
    \epsilon_{zz}   \\
    2 \epsilon_{xy} \\
    2 \epsilon_{xz} \\
    2 \epsilon_{yz}
  \end{bmatrix} = \begin{bmatrix}
    \displaystyle \frac{ \partial u_x }{ \partial x }                                       \\
    \displaystyle \frac{ \partial u_y }{ \partial y }                                       \\
    \displaystyle \frac{ \partial u_z }{ \partial z }                                       \\
    \displaystyle \frac{ \partial u_x }{ \partial y } + \frac{ \partial u_y }{ \partial x } \\
    \displaystyle \frac{ \partial u_x }{ \partial z } + \frac{ \partial u_z }{ \partial x } \\
    \displaystyle \frac{ \partial u_y }{ \partial z } + \frac{ \partial u_z }{ \partial y }
  \end{bmatrix} = \begin{bmatrix}
    \displaystyle \frac{ \partial  }{ \partial x } & \displaystyle 0                                & \displaystyle 0                                \\
    \displaystyle 0                                & \displaystyle \frac{ \partial  }{ \partial y } & \displaystyle 0                                \\
    \displaystyle 0                                & \displaystyle 0                                & \displaystyle \frac{ \partial  }{ \partial z } \\
    \displaystyle \frac{ \partial  }{ \partial y } & \displaystyle \frac{ \partial  }{ \partial x } & \displaystyle 0                                \\
    \displaystyle \frac{ \partial  }{ \partial z } & \displaystyle 0                                & \displaystyle \frac{ \partial  }{ \partial x } \\
    \displaystyle 0                                & \displaystyle \frac{ \partial  }{ \partial z } & \displaystyle \frac{ \partial  }{ \partial y }
  \end{bmatrix} \begin{bmatrix}
    u_x \\
    u_y \\
    u_z
  \end{bmatrix} = \boldsymbol S \boldsymbol u.
\end{equation}
%
\subsection{Step 3: Make an approximation}
Now we make the finite element approximation
%
\begin{equation}
  \boldsymbol u \approx \boldsymbol N^e \boldsymbol u^e = \sum_\alpha N_\alpha^e(\boldsymbol X) \,u_\alpha^e
\end{equation}
%
where $\alpha = i,j,k,\dots$ are the corners of the mesh element $e$ of
the 3D mesh. Written out fulle, it becomes
%
\begin{align}
  \boldsymbol u \approx
  \begin{bmatrix}
    N_{i_x}^e \hspace{-6pt}& 0       \hspace{-6pt}& 0       \hspace{-6pt}& N_{j_x}^e \hspace{-6pt}& 0         \hspace{-6pt}& 0         \hspace{-6pt}& N_{k_x}^e \hspace{-6pt}& 0         \hspace{-6pt}& 0         \\
    0         \hspace{-6pt}& N_{i_y} \hspace{-6pt}& 0       \hspace{-6pt}& 0         \hspace{-6pt}& N_{j_y}^e \hspace{-6pt}& 0         \hspace{-6pt}& 0         \hspace{-6pt}& N_{k_y}^e \hspace{-6pt}& 0         \\
    0         \hspace{-6pt}& 0       \hspace{-6pt}& N_{i_z} \hspace{-6pt}& 0         \hspace{-6pt}& 0         \hspace{-6pt}& N_{j_z}^e \hspace{-6pt}& 0         \hspace{-6pt}& 0         \hspace{-6pt}& N_{k_z}^e
  \end{bmatrix}
  \begin{bmatrix}
    u_{i_x}^e \\
    u_{i_y}^e \\
    u_{i_z}^e \\
    u_{j_x}^e \\
    u_{j_y}^e \\
    u_{j_z}^e \\
    u_{k_x}^e \\
    u_{k_y}^e \\
    u_{k_z}^e
  \end{bmatrix}
\end{align}
%
\subsubsection{Choose a trial function}
We again use the \textit{Galerkin method}  for our trial functions
%
\begin{equation}
  \boldsymbol w = \boldsymbol N^e \boldsymbol \delta\boldsymbol w^e.
\end{equation}
%
Given our weak form
%
\begin{equation}
  \int_{\Omega_{\boldsymbol x}} \boldsymbol \sigma : \boldsymbol \epsilon_{\boldsymbol w} \, d\Omega_{\boldsymbol x}- \int_{\Gamma_{\boldsymbol t}} \boldsymbol w^T \boldsymbol t \, d\Gamma_{\boldsymbol t}- \int_{\Omega_{\boldsymbol x}} \boldsymbol w^T \boldsymbol b \, d\Omega_{\boldsymbol x} = \boldsymbol 0,
\end{equation}
%
we know from above that $\boldsymbol \sigma = \boldsymbol D \boldsymbol
  \epsilon$ and $\boldsymbol \epsilon = \boldsymbol S \boldsymbol u \approx
  \boldsymbol S \boldsymbol N^e \boldsymbol u^e = \boldsymbol B \boldsymbol u^e$,
where we've defined $\boldsymbol B = \boldsymbol S \boldsymbol N^e$. Now we can
rewrite
%
\begin{equation}
  \boldsymbol \sigma : \boldsymbol \epsilon_{\boldsymbol w} = \boldsymbol \sigma^T \boldsymbol \epsilon_{\boldsymbol w} = \boldsymbol \epsilon^T \boldsymbol D \boldsymbol \epsilon = \boldsymbol \delta{\boldsymbol w^e}^T {\boldsymbol B^e}^T \boldsymbol D \boldsymbol B^e \boldsymbol u^e,
\end{equation}
%
where after the first equals, $\boldsymbol \sigma$ and $\boldsymbol
  \epsilon_{\boldsymbol w}$ are in vector form. Now we can insert into the weak
form and get
%
\begin{align}
  \left( \int_{\Omega_{\boldsymbol x}} \boldsymbol \delta{\boldsymbol w^e}^T {\boldsymbol B^e}^T \boldsymbol D \, \boldsymbol B^e \, d\Omega_{\boldsymbol x} \right) \boldsymbol u^e
  &- \int_{\Gamma_{\boldsymbol t}} \boldsymbol \delta{\boldsymbol w^e}^T {\boldsymbol N^e}^T  \, \boldsymbol t \, d\Gamma_{\boldsymbol t} \notag \\
  & \quad - \int_{\Omega_{\boldsymbol x}} \boldsymbol \delta{\boldsymbol w^e}^T {\boldsymbol N^e}^T \boldsymbol b \, d\Omega_{\boldsymbol x} = \boldsymbol 0,
\end{align}
%
which is equivalent to
%
\begin{align}
  \boldsymbol \delta{\boldsymbol w^e}^T\Bigg[\left( \int_{\Omega_{\boldsymbol x}} {\boldsymbol B^e}^T \boldsymbol D \, \boldsymbol B^e \, d\Omega_{\boldsymbol x} \right) \boldsymbol u^e
  & - \int_{\Gamma_{\boldsymbol t}} {\boldsymbol N^e}^T  \, \boldsymbol t \, d\Gamma_{\boldsymbol t} \notag \\
  & \qquad - \int_{\Omega_{\boldsymbol x}} {\boldsymbol N^e}^T \boldsymbol b \, d\Omega_{\boldsymbol x} \Bigg] = \boldsymbol 0.
\end{align}
%
Now since the variables $\boldsymbol \delta \boldsymbol w^e$ are chosen to be
random, the term inside the square brackets must be equal to zero. Also we can
suffice with integrating only elementwise now (see integration bounds)
%
\begin{equation}
  \left( \int_{\Omega_{\boldsymbol x}^e} {\boldsymbol B^e}^T \boldsymbol D \, \boldsymbol B^e \, d\Omega_{\boldsymbol x} \right) \boldsymbol u^e
- \int_{\Gamma_{\boldsymbol t}^e} {\boldsymbol N^e}^T  \, \boldsymbol t \, d\Gamma_{\boldsymbol t}
- \int_{\Omega_{\boldsymbol x}^e} {\boldsymbol N^e}^T \boldsymbol b \, d\Omega_{\boldsymbol x} = \boldsymbol 0.
\end{equation}
%
Now, since the second and third terms don't depend on our displacement field
$\boldsymbol u$, we can solve for these, using known values. Additionally, if
we the shape functions $\boldsymbol N$ are of first order, we know that the
matrix $\boldsymbol B$ is constant, and since $\boldsymbol D$ is constant per
definition, we can also simplify the first term. We get
%
\begin{equation}
  \left( {\boldsymbol B^e}^T \boldsymbol D \, \boldsymbol B^e \int_{\Omega_{\boldsymbol x}^e} \, d\Omega_{\boldsymbol x} \right) \boldsymbol u^e + \mathbf f_t^e + \mathbf f_b^e  = \boldsymbol 0,
\end{equation}
%
which is the same as
%
\begin{align}
  V^e \, {\boldsymbol B^e}^T \boldsymbol D \boldsymbol B^e \,\boldsymbol u^e + \mathbf f ^e & = \boldsymbol 0, \\
  \boldsymbol K^e \boldsymbol u^e + \mathbf f^e                                             & = \boldsymbol 0.
\end{align}
%
And we're done! Now we can solve this linear system using whatever numerical
solver we'd like.

\subsection{Apply FEM to a more complex problem such as linear elasticity.}
\subsection{Apply experimental validation to a FEM implementation of for instance linear
      elasticity.}

\end{document}
\endinput
